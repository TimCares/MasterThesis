
\documentclass[10pt]{article} % For LaTeX2e
\usepackage{tmlr}
% If accepted, instead use the following line for the camera-ready submission:
%\usepackage[accepted]{tmlr}
% To de-anonymize and remove mentions to TMLR (for example for posting to preprint servers), instead use the following:
%\usepackage[preprint]{tmlr}

% Optional math commands from https://github.com/goodfeli/dlbook_notation.
%%%%% NEW MATH DEFINITIONS %%%%%

\usepackage{amsmath,amsfonts,bm}

% Mark sections of captions for referring to divisions of figures
\newcommand{\figleft}{{\em (Left)}}
\newcommand{\figcenter}{{\em (Center)}}
\newcommand{\figright}{{\em (Right)}}
\newcommand{\figtop}{{\em (Top)}}
\newcommand{\figbottom}{{\em (Bottom)}}
\newcommand{\captiona}{{\em (a)}}
\newcommand{\captionb}{{\em (b)}}
\newcommand{\captionc}{{\em (c)}}
\newcommand{\captiond}{{\em (d)}}

% Highlight a newly defined term
\newcommand{\newterm}[1]{{\bf #1}}


% Figure reference, lower-case.
\def\figref#1{figure~\ref{#1}}
% Figure reference, capital. For start of sentence
\def\Figref#1{Figure~\ref{#1}}
\def\twofigref#1#2{figures \ref{#1} and \ref{#2}}
\def\quadfigref#1#2#3#4{figures \ref{#1}, \ref{#2}, \ref{#3} and \ref{#4}}
% Section reference, lower-case.
\def\secref#1{section~\ref{#1}}
% Section reference, capital.
\def\Secref#1{Section~\ref{#1}}
% Reference to two sections.
\def\twosecrefs#1#2{sections \ref{#1} and \ref{#2}}
% Reference to three sections.
\def\secrefs#1#2#3{sections \ref{#1}, \ref{#2} and \ref{#3}}
% Reference to an equation, lower-case.
\def\eqref#1{equation~\ref{#1}}
% Reference to an equation, upper case
\def\Eqref#1{Equation~\ref{#1}}
% A raw reference to an equation---avoid using if possible
\def\plaineqref#1{\ref{#1}}
% Reference to a chapter, lower-case.
\def\chapref#1{chapter~\ref{#1}}
% Reference to an equation, upper case.
\def\Chapref#1{Chapter~\ref{#1}}
% Reference to a range of chapters
\def\rangechapref#1#2{chapters\ref{#1}--\ref{#2}}
% Reference to an algorithm, lower-case.
\def\algref#1{algorithm~\ref{#1}}
% Reference to an algorithm, upper case.
\def\Algref#1{Algorithm~\ref{#1}}
\def\twoalgref#1#2{algorithms \ref{#1} and \ref{#2}}
\def\Twoalgref#1#2{Algorithms \ref{#1} and \ref{#2}}
% Reference to a part, lower case
\def\partref#1{part~\ref{#1}}
% Reference to a part, upper case
\def\Partref#1{Part~\ref{#1}}
\def\twopartref#1#2{parts \ref{#1} and \ref{#2}}

\def\ceil#1{\lceil #1 \rceil}
\def\floor#1{\lfloor #1 \rfloor}
\def\1{\bm{1}}
\newcommand{\train}{\mathcal{D}}
\newcommand{\valid}{\mathcal{D_{\mathrm{valid}}}}
\newcommand{\test}{\mathcal{D_{\mathrm{test}}}}

\def\eps{{\epsilon}}


% Random variables
\def\reta{{\textnormal{$\eta$}}}
\def\ra{{\textnormal{a}}}
\def\rb{{\textnormal{b}}}
\def\rc{{\textnormal{c}}}
\def\rd{{\textnormal{d}}}
\def\re{{\textnormal{e}}}
\def\rf{{\textnormal{f}}}
\def\rg{{\textnormal{g}}}
\def\rh{{\textnormal{h}}}
\def\ri{{\textnormal{i}}}
\def\rj{{\textnormal{j}}}
\def\rk{{\textnormal{k}}}
\def\rl{{\textnormal{l}}}
% rm is already a command, just don't name any random variables m
\def\rn{{\textnormal{n}}}
\def\ro{{\textnormal{o}}}
\def\rp{{\textnormal{p}}}
\def\rq{{\textnormal{q}}}
\def\rr{{\textnormal{r}}}
\def\rs{{\textnormal{s}}}
\def\rt{{\textnormal{t}}}
\def\ru{{\textnormal{u}}}
\def\rv{{\textnormal{v}}}
\def\rw{{\textnormal{w}}}
\def\rx{{\textnormal{x}}}
\def\ry{{\textnormal{y}}}
\def\rz{{\textnormal{z}}}

% Random vectors
\def\rvepsilon{{\mathbf{\epsilon}}}
\def\rvtheta{{\mathbf{\theta}}}
\def\rva{{\mathbf{a}}}
\def\rvb{{\mathbf{b}}}
\def\rvc{{\mathbf{c}}}
\def\rvd{{\mathbf{d}}}
\def\rve{{\mathbf{e}}}
\def\rvf{{\mathbf{f}}}
\def\rvg{{\mathbf{g}}}
\def\rvh{{\mathbf{h}}}
\def\rvu{{\mathbf{i}}}
\def\rvj{{\mathbf{j}}}
\def\rvk{{\mathbf{k}}}
\def\rvl{{\mathbf{l}}}
\def\rvm{{\mathbf{m}}}
\def\rvn{{\mathbf{n}}}
\def\rvo{{\mathbf{o}}}
\def\rvp{{\mathbf{p}}}
\def\rvq{{\mathbf{q}}}
\def\rvr{{\mathbf{r}}}
\def\rvs{{\mathbf{s}}}
\def\rvt{{\mathbf{t}}}
\def\rvu{{\mathbf{u}}}
\def\rvv{{\mathbf{v}}}
\def\rvw{{\mathbf{w}}}
\def\rvx{{\mathbf{x}}}
\def\rvy{{\mathbf{y}}}
\def\rvz{{\mathbf{z}}}

% Elements of random vectors
\def\erva{{\textnormal{a}}}
\def\ervb{{\textnormal{b}}}
\def\ervc{{\textnormal{c}}}
\def\ervd{{\textnormal{d}}}
\def\erve{{\textnormal{e}}}
\def\ervf{{\textnormal{f}}}
\def\ervg{{\textnormal{g}}}
\def\ervh{{\textnormal{h}}}
\def\ervi{{\textnormal{i}}}
\def\ervj{{\textnormal{j}}}
\def\ervk{{\textnormal{k}}}
\def\ervl{{\textnormal{l}}}
\def\ervm{{\textnormal{m}}}
\def\ervn{{\textnormal{n}}}
\def\ervo{{\textnormal{o}}}
\def\ervp{{\textnormal{p}}}
\def\ervq{{\textnormal{q}}}
\def\ervr{{\textnormal{r}}}
\def\ervs{{\textnormal{s}}}
\def\ervt{{\textnormal{t}}}
\def\ervu{{\textnormal{u}}}
\def\ervv{{\textnormal{v}}}
\def\ervw{{\textnormal{w}}}
\def\ervx{{\textnormal{x}}}
\def\ervy{{\textnormal{y}}}
\def\ervz{{\textnormal{z}}}

% Random matrices
\def\rmA{{\mathbf{A}}}
\def\rmB{{\mathbf{B}}}
\def\rmC{{\mathbf{C}}}
\def\rmD{{\mathbf{D}}}
\def\rmE{{\mathbf{E}}}
\def\rmF{{\mathbf{F}}}
\def\rmG{{\mathbf{G}}}
\def\rmH{{\mathbf{H}}}
\def\rmI{{\mathbf{I}}}
\def\rmJ{{\mathbf{J}}}
\def\rmK{{\mathbf{K}}}
\def\rmL{{\mathbf{L}}}
\def\rmM{{\mathbf{M}}}
\def\rmN{{\mathbf{N}}}
\def\rmO{{\mathbf{O}}}
\def\rmP{{\mathbf{P}}}
\def\rmQ{{\mathbf{Q}}}
\def\rmR{{\mathbf{R}}}
\def\rmS{{\mathbf{S}}}
\def\rmT{{\mathbf{T}}}
\def\rmU{{\mathbf{U}}}
\def\rmV{{\mathbf{V}}}
\def\rmW{{\mathbf{W}}}
\def\rmX{{\mathbf{X}}}
\def\rmY{{\mathbf{Y}}}
\def\rmZ{{\mathbf{Z}}}

% Elements of random matrices
\def\ermA{{\textnormal{A}}}
\def\ermB{{\textnormal{B}}}
\def\ermC{{\textnormal{C}}}
\def\ermD{{\textnormal{D}}}
\def\ermE{{\textnormal{E}}}
\def\ermF{{\textnormal{F}}}
\def\ermG{{\textnormal{G}}}
\def\ermH{{\textnormal{H}}}
\def\ermI{{\textnormal{I}}}
\def\ermJ{{\textnormal{J}}}
\def\ermK{{\textnormal{K}}}
\def\ermL{{\textnormal{L}}}
\def\ermM{{\textnormal{M}}}
\def\ermN{{\textnormal{N}}}
\def\ermO{{\textnormal{O}}}
\def\ermP{{\textnormal{P}}}
\def\ermQ{{\textnormal{Q}}}
\def\ermR{{\textnormal{R}}}
\def\ermS{{\textnormal{S}}}
\def\ermT{{\textnormal{T}}}
\def\ermU{{\textnormal{U}}}
\def\ermV{{\textnormal{V}}}
\def\ermW{{\textnormal{W}}}
\def\ermX{{\textnormal{X}}}
\def\ermY{{\textnormal{Y}}}
\def\ermZ{{\textnormal{Z}}}

% Vectors
\def\vzero{{\bm{0}}}
\def\vone{{\bm{1}}}
\def\vmu{{\bm{\mu}}}
\def\vtheta{{\bm{\theta}}}
\def\va{{\bm{a}}}
\def\vb{{\bm{b}}}
\def\vc{{\bm{c}}}
\def\vd{{\bm{d}}}
\def\ve{{\bm{e}}}
\def\vf{{\bm{f}}}
\def\vg{{\bm{g}}}
\def\vh{{\bm{h}}}
\def\vi{{\bm{i}}}
\def\vj{{\bm{j}}}
\def\vk{{\bm{k}}}
\def\vl{{\bm{l}}}
\def\vm{{\bm{m}}}
\def\vn{{\bm{n}}}
\def\vo{{\bm{o}}}
\def\vp{{\bm{p}}}
\def\vq{{\bm{q}}}
\def\vr{{\bm{r}}}
\def\vs{{\bm{s}}}
\def\vt{{\bm{t}}}
\def\vu{{\bm{u}}}
\def\vv{{\bm{v}}}
\def\vw{{\bm{w}}}
\def\vx{{\bm{x}}}
\def\vy{{\bm{y}}}
\def\vz{{\bm{z}}}

% Elements of vectors
\def\evalpha{{\alpha}}
\def\evbeta{{\beta}}
\def\evepsilon{{\epsilon}}
\def\evlambda{{\lambda}}
\def\evomega{{\omega}}
\def\evmu{{\mu}}
\def\evpsi{{\psi}}
\def\evsigma{{\sigma}}
\def\evtheta{{\theta}}
\def\eva{{a}}
\def\evb{{b}}
\def\evc{{c}}
\def\evd{{d}}
\def\eve{{e}}
\def\evf{{f}}
\def\evg{{g}}
\def\evh{{h}}
\def\evi{{i}}
\def\evj{{j}}
\def\evk{{k}}
\def\evl{{l}}
\def\evm{{m}}
\def\evn{{n}}
\def\evo{{o}}
\def\evp{{p}}
\def\evq{{q}}
\def\evr{{r}}
\def\evs{{s}}
\def\evt{{t}}
\def\evu{{u}}
\def\evv{{v}}
\def\evw{{w}}
\def\evx{{x}}
\def\evy{{y}}
\def\evz{{z}}

% Matrix
\def\mA{{\bm{A}}}
\def\mB{{\bm{B}}}
\def\mC{{\bm{C}}}
\def\mD{{\bm{D}}}
\def\mE{{\bm{E}}}
\def\mF{{\bm{F}}}
\def\mG{{\bm{G}}}
\def\mH{{\bm{H}}}
\def\mI{{\bm{I}}}
\def\mJ{{\bm{J}}}
\def\mK{{\bm{K}}}
\def\mL{{\bm{L}}}
\def\mM{{\bm{M}}}
\def\mN{{\bm{N}}}
\def\mO{{\bm{O}}}
\def\mP{{\bm{P}}}
\def\mQ{{\bm{Q}}}
\def\mR{{\bm{R}}}
\def\mS{{\bm{S}}}
\def\mT{{\bm{T}}}
\def\mU{{\bm{U}}}
\def\mV{{\bm{V}}}
\def\mW{{\bm{W}}}
\def\mX{{\bm{X}}}
\def\mY{{\bm{Y}}}
\def\mZ{{\bm{Z}}}
\def\mBeta{{\bm{\beta}}}
\def\mPhi{{\bm{\Phi}}}
\def\mLambda{{\bm{\Lambda}}}
\def\mSigma{{\bm{\Sigma}}}

% Tensor
\DeclareMathAlphabet{\mathsfit}{\encodingdefault}{\sfdefault}{m}{sl}
\SetMathAlphabet{\mathsfit}{bold}{\encodingdefault}{\sfdefault}{bx}{n}
\newcommand{\tens}[1]{\bm{\mathsfit{#1}}}
\def\tA{{\tens{A}}}
\def\tB{{\tens{B}}}
\def\tC{{\tens{C}}}
\def\tD{{\tens{D}}}
\def\tE{{\tens{E}}}
\def\tF{{\tens{F}}}
\def\tG{{\tens{G}}}
\def\tH{{\tens{H}}}
\def\tI{{\tens{I}}}
\def\tJ{{\tens{J}}}
\def\tK{{\tens{K}}}
\def\tL{{\tens{L}}}
\def\tM{{\tens{M}}}
\def\tN{{\tens{N}}}
\def\tO{{\tens{O}}}
\def\tP{{\tens{P}}}
\def\tQ{{\tens{Q}}}
\def\tR{{\tens{R}}}
\def\tS{{\tens{S}}}
\def\tT{{\tens{T}}}
\def\tU{{\tens{U}}}
\def\tV{{\tens{V}}}
\def\tW{{\tens{W}}}
\def\tX{{\tens{X}}}
\def\tY{{\tens{Y}}}
\def\tZ{{\tens{Z}}}


% Graph
\def\gA{{\mathcal{A}}}
\def\gB{{\mathcal{B}}}
\def\gC{{\mathcal{C}}}
\def\gD{{\mathcal{D}}}
\def\gE{{\mathcal{E}}}
\def\gF{{\mathcal{F}}}
\def\gG{{\mathcal{G}}}
\def\gH{{\mathcal{H}}}
\def\gI{{\mathcal{I}}}
\def\gJ{{\mathcal{J}}}
\def\gK{{\mathcal{K}}}
\def\gL{{\mathcal{L}}}
\def\gM{{\mathcal{M}}}
\def\gN{{\mathcal{N}}}
\def\gO{{\mathcal{O}}}
\def\gP{{\mathcal{P}}}
\def\gQ{{\mathcal{Q}}}
\def\gR{{\mathcal{R}}}
\def\gS{{\mathcal{S}}}
\def\gT{{\mathcal{T}}}
\def\gU{{\mathcal{U}}}
\def\gV{{\mathcal{V}}}
\def\gW{{\mathcal{W}}}
\def\gX{{\mathcal{X}}}
\def\gY{{\mathcal{Y}}}
\def\gZ{{\mathcal{Z}}}

% Sets
\def\sA{{\mathbb{A}}}
\def\sB{{\mathbb{B}}}
\def\sC{{\mathbb{C}}}
\def\sD{{\mathbb{D}}}
% Don't use a set called E, because this would be the same as our symbol
% for expectation.
\def\sF{{\mathbb{F}}}
\def\sG{{\mathbb{G}}}
\def\sH{{\mathbb{H}}}
\def\sI{{\mathbb{I}}}
\def\sJ{{\mathbb{J}}}
\def\sK{{\mathbb{K}}}
\def\sL{{\mathbb{L}}}
\def\sM{{\mathbb{M}}}
\def\sN{{\mathbb{N}}}
\def\sO{{\mathbb{O}}}
\def\sP{{\mathbb{P}}}
\def\sQ{{\mathbb{Q}}}
\def\sR{{\mathbb{R}}}
\def\sS{{\mathbb{S}}}
\def\sT{{\mathbb{T}}}
\def\sU{{\mathbb{U}}}
\def\sV{{\mathbb{V}}}
\def\sW{{\mathbb{W}}}
\def\sX{{\mathbb{X}}}
\def\sY{{\mathbb{Y}}}
\def\sZ{{\mathbb{Z}}}

% Entries of a matrix
\def\emLambda{{\Lambda}}
\def\emA{{A}}
\def\emB{{B}}
\def\emC{{C}}
\def\emD{{D}}
\def\emE{{E}}
\def\emF{{F}}
\def\emG{{G}}
\def\emH{{H}}
\def\emI{{I}}
\def\emJ{{J}}
\def\emK{{K}}
\def\emL{{L}}
\def\emM{{M}}
\def\emN{{N}}
\def\emO{{O}}
\def\emP{{P}}
\def\emQ{{Q}}
\def\emR{{R}}
\def\emS{{S}}
\def\emT{{T}}
\def\emU{{U}}
\def\emV{{V}}
\def\emW{{W}}
\def\emX{{X}}
\def\emY{{Y}}
\def\emZ{{Z}}
\def\emSigma{{\Sigma}}

% entries of a tensor
% Same font as tensor, without \bm wrapper
\newcommand{\etens}[1]{\mathsfit{#1}}
\def\etLambda{{\etens{\Lambda}}}
\def\etA{{\etens{A}}}
\def\etB{{\etens{B}}}
\def\etC{{\etens{C}}}
\def\etD{{\etens{D}}}
\def\etE{{\etens{E}}}
\def\etF{{\etens{F}}}
\def\etG{{\etens{G}}}
\def\etH{{\etens{H}}}
\def\etI{{\etens{I}}}
\def\etJ{{\etens{J}}}
\def\etK{{\etens{K}}}
\def\etL{{\etens{L}}}
\def\etM{{\etens{M}}}
\def\etN{{\etens{N}}}
\def\etO{{\etens{O}}}
\def\etP{{\etens{P}}}
\def\etQ{{\etens{Q}}}
\def\etR{{\etens{R}}}
\def\etS{{\etens{S}}}
\def\etT{{\etens{T}}}
\def\etU{{\etens{U}}}
\def\etV{{\etens{V}}}
\def\etW{{\etens{W}}}
\def\etX{{\etens{X}}}
\def\etY{{\etens{Y}}}
\def\etZ{{\etens{Z}}}

% The true underlying data generating distribution
\newcommand{\pdata}{p_{\rm{data}}}
% The empirical distribution defined by the training set
\newcommand{\ptrain}{\hat{p}_{\rm{data}}}
\newcommand{\Ptrain}{\hat{P}_{\rm{data}}}
% The model distribution
\newcommand{\pmodel}{p_{\rm{model}}}
\newcommand{\Pmodel}{P_{\rm{model}}}
\newcommand{\ptildemodel}{\tilde{p}_{\rm{model}}}
% Stochastic autoencoder distributions
\newcommand{\pencode}{p_{\rm{encoder}}}
\newcommand{\pdecode}{p_{\rm{decoder}}}
\newcommand{\precons}{p_{\rm{reconstruct}}}

\newcommand{\laplace}{\mathrm{Laplace}} % Laplace distribution

\newcommand{\E}{\mathbb{E}}
\newcommand{\Ls}{\mathcal{L}}
\newcommand{\R}{\mathbb{R}}
\newcommand{\emp}{\tilde{p}}
\newcommand{\lr}{\alpha}
\newcommand{\reg}{\lambda}
\newcommand{\rect}{\mathrm{rectifier}}
\newcommand{\softmax}{\mathrm{softmax}}
\newcommand{\sigmoid}{\sigma}
\newcommand{\softplus}{\zeta}
\newcommand{\KL}{D_{\mathrm{KL}}}
\newcommand{\Var}{\mathrm{Var}}
\newcommand{\standarderror}{\mathrm{SE}}
\newcommand{\Cov}{\mathrm{Cov}}
% Wolfram Mathworld says $L^2$ is for function spaces and $\ell^2$ is for vectors
% But then they seem to use $L^2$ for vectors throughout the site, and so does
% wikipedia.
\newcommand{\normlzero}{L^0}
\newcommand{\normlone}{L^1}
\newcommand{\normltwo}{L^2}
\newcommand{\normlp}{L^p}
\newcommand{\normmax}{L^\infty}

\newcommand{\parents}{Pa} % See usage in notation.tex. Chosen to match Daphne's book.

\DeclareMathOperator*{\argmax}{arg\,max}
\DeclareMathOperator*{\argmin}{arg\,min}

\DeclareMathOperator{\sign}{sign}
\DeclareMathOperator{\Tr}{Tr}
\let\ab\allowbreak


\usepackage{hyperref}
\usepackage{url}


\title{Leveraging pretrained unimodal models for efficient image-text retrieval}

% Authors must not appear in the submitted version. They should be hidden
% as long as the tmlr package is used without the [accepted] or [preprint] options.
% Non-anonymous submissions will be rejected without review.

% \author{\name Tim Cares \email tim.cares@icloud.com \\
%      \addr Department of Economics and Computer Science\\
%      University of Hannover}

% The \author macro works with any number of authors. Use \AND 
% to separate the names and addresses of multiple authors.

\newcommand{\fix}{\marginpar{FIX}}
\newcommand{\new}{\marginpar{NEW}}

\def\month{MM}  % Insert correct month for camera-ready version
\def\year{YYYY} % Insert correct year for camera-ready version
\def\openreview{\url{https://openreview.net/forum?id=XXXX}} % Insert correct link to OpenReview for camera-ready version


\begin{document}


\maketitle

\begin{abstract}
Multimodal models, especially vision-language models, have gained increasing popularity due to their wide range of applications,
and show impressive performance especially on retrieval tasks. However, existing approaches often require large-scale models,
extensive data, and substantial computational resources, limiting their accessibility for smaller research groups and individuals.
We address this issue by introducing an efficient self-supervised vision-language model for image-text retrieval that is significantly
cheaper to train and smaller in size. We leverage pretrained unimodal encoders and introduce a randomly initialized shared encoder to align
representations using a contrastive loss function. A self-supervised image model is employed for simultaneous knowledge distillation,
guiding the alignment through high-level image representations. While not reaching SOTA performance, our approach demonstrates competitive
performance with popular vision-language models like CLIP and FLAVA on retrieval tasks, outperforming them on certain metrics while using
only 0.75\% of the data used by CLIP and 4.3\% by FLAVA. These findings underscore the potential for designing efficient multimodal retrieval
systems, and therefore lay the foundation for future research on financially accessible models, promoting broader participation in multimodal
learning. To promote transparency and facilitate further research, we have made our code for training and evaluating our model publicly available.
\end{abstract}

\section{Introduction}
Existing vision-language models have seen a significant increase in parameter count and training data, namely image-text pairs.
Along with emerging pretraining objectives like a large-scale contrastive loss (cite clip, coca, vlmo) and especially masked
vision-language modeling (cite beit-3, flava) those models have reached near perfect score on the widely used benchmarks MSCOCO
(cite) and Flickr30K (cite) for image-text retrieval. While there is the risk that samples from these benchmarks may end up in
the training data of those approaches, due to their large-scale training datasets, their ability to connect real-world concepts
across image and text remains remarkable.

However, with an increase in parameters and training data, the resources (mainly costs through accelerators) to train these models
can only be covered by large companies. For example CLIP (cite) has been trained on 400 million image-text pairs, and the largest
model has 428 (cite huggingface) million parameters. Based on our estimate (footnote) a reproduction of this model would cost more
than 77 thousand dollars to train. For approaches where we are able to estimate the costs based on the information published by
the authors, we observe a similar trend: VLMo (cite) costs more than 9 thousand dollars to train, and CoCa (cite) even more than
350 thousand dollars.

In this paper, we propose a method similar to that of \citet{shre}, and leverage pretrained unimodal models to 

Our contributions are as follows:
\begin{itemize}
\item We show that using pretrained image and text components can reduce the training costs for image-text retrieval models dramatically.
\item We demonstrate that a contrastive loss with a low batch size yields a surprising good performance.
\item Using a self-supervised vision model as the teacher for knowledge distillation leads to better performance than a supervised vision model.
\item An approach characterized by a fully end-to-end self-supervised training on uncurated image-text data.
\end{itemize}

\section{Related work}

\textbf{Knowledge Distillation for guidance.} This paper is motivated by the work of \citet{shre}, which train a multimodal model
for the alignment of image, text, and audio. The authors use a supervised vision model as a teacher, which provides a probability
distribution over the ImageNet-1K \citep{imagenet} classes. Because \citet{shre} use image-text and image-audio pairs, the multimodal
(student) model can predict the probability distribution over the ImageNet-1K \citep{imagenet} classes when receiving the same image
as the teacher, and most importantly the text and audio of the image-text and image-audio pair respectively. The intuition is that since
image and text (or image and audio) contain the same semantic content, the ImageNet-1K \citep{imagenet} classes of the image should also
describe the content of the corresponding text (audio). An example of this (for the paper relevant) image-text pairs can be seen in TODO
in the Appendix. Predicting the probability distribution for an image, and an additional ranking loss, leads to an alignment between image,
text, and audio, which can be exploited to perform cross-modal retrieval.

\textbf{Contrastive learning for image-text alignment.}
OpenAI's CLIP \citep{clip} was the first model which exclusively relied on a large scale contrastive loss to align image and text.
The authors showed that with sufficient amount of data and a large batch size the contrastive loss leads to a strong alignment
between image and text. This lead to a widespread adoption of contrastive learning with large batch sizes in vision-language
pretraining, and has become the de-facto standard to align image and text (\citet{coca}; \citet{vlmo}; \citet{flava}; \citet{filip}),
and has only recently been shown to not be essential for models upwards of a billion parameters \citep{beit3}.

\textbf{Bootstrapping by pretrained initialization.} A well-known practice is to use the weights of models (pre-)trained on tasks
similar to the target tasks to reduce data requirements and speed up convergence. Since vision-language models usually have parameters
exclusively responsible for image and text, it makes sense to initialize these parts of the model with weights from pretrained image and
text models, respectively. This is a practice adopted by \cite{vlmo} and \cite{flava}, and both approaches showed significant improvements
compared to a random initialization. While the selection for pretrained language models to initialize the text components of the vision-language
model naturally falls to self-supervised trained language models like BERT \citep{bert}, since masked language modeling leads to a strong
understanding of text, one should proceed with care when selecting the right vision model for initialization. It is tempting to use
supervised vision models, as they still lead to superior performance compared to self-supervised vision models. However, when using
a vision model trained with labeled data the end-to-end process is not fully self-supervised anymore, and can therefore be considered as
"cheating". It is because of this that using only self-supervised components for the initialization is essential.

\section{Method}
Our method is characterized by three main concepts: self-supervised knowledge distillation, contrastive learning, and the initialization. All
three concepts are, in order, inspired by the related works presented in the previous section. Before we present the details of our method,
we first establish criteria our approach fulfills and why we believe these criteria are important.
\subsection{Criteria}
\textbf{End-to-end self-supervised.} We believe that a fully self-supervised training process is essential, as this (1) allows to scale up our
method if desired (even though this is not the focus of this paper), because we do not rely on labeled data, and (2) we can perform a fair
comparison with existing approaches to image-text retrieval. The latter is important because, as already mentioned in the previous section,
using supervised models for initialization can be considered as cheating. Even if our training process is self-supervised, the use of
(pre-)trained supervised components for initialization turns the whole (end-to-end) process into a supervised one. Whether image-text pairs
can be considered as labeled data is a matter of debate, and we discuss this in Appendix \ref{curated_data}.

\textbf{Independence to pretrained vision-language components.} Perhaps the most important criterion is that our method is independent of
pretrained vision-language components. We build our method as if the paradigm of vision-language models does not exist, and only rely on
pretrained unimodal models. This is important because it allows for a fair comparison with existing approaches, and our results would otherwise
most likely be the result of the pretrained vision-language components. Again, this can be considered as cheating and would drastically reduce
the significance of our results.

\textbf{Efficiency.} The primary goal of our method is to reduce the costs of (pre-)training image-text retrieval models. Therefore, an
obvious criterion is that our method is efficient in terms of parameter count, training data, and computational resources. It follows that
our method should be significantly cheaper to train than existing approaches.

\textbf{Performance.} While the primary goal of our method is to reduce the costs of training image-text retrieval models, we still aim to
achieve competitive performance with existing approaches, for example CLIP \citep{clip}. However, since this work is \textbf{fully self-funded}
and not backed by a large enterprise or research institution, we do not aim to reach state-of-the-art performance. Instead, the goal is to
demonstrate that it is possible to achieve somewhat competitive performance with a fraction of the costs. What will come apparent when we
present our results is that we neither reach the state-of-the-art performance on MSCOCO \citep{coco} and Flickr30K \citep{flickr30k}
retrieval, as currently\footnote{As of September 2024.} held by BEiT-3 \citep{beit3}, nor do we aim to do so.

\subsection{Architecture and Initialization}
Our vision-language model consists of three components: a pretrained image encoder, a pretrained text encoder, and a randomly initialized
shared encoder.
The latter has to be randomly initialized to fulfill the criterion of independence to pretrained vision-language components.
As their name suggests, the image encoder is responsible for encoding images, and the text encoder is responsible for encoding text.
Therefore, they are specific to their respective modality, and we can therefore initialize them with pretrained unimodal models.
For the image encoder, we use the pretrained Data2Vec2 \citep{data2vec2} image model, and for the text encoder, we use the pretrained
BERT base \citep{bert} model. Since each of these models is a 12-layer Transformer \citep{transformer}, which already has 86 million
parameters, we only take the first 6 layers of each model to reduce the parameter count. Using other strategies like every second layer
(e.g. 1, 3, 5, ...) leads to a worse performance in preliminary experiments. Note that both models have been trained self-supervised on
image and text data, respectively, and therefore fulfill our criteria defined in the previous section. To keep the parameter count manageable,
the shared encoder is merely a single Transformer layer and follows the ViT \citep{vit} architecture. An overview of the architecture can
be seen in Figure TODO.

\textbf{Text Representation.} Each caption/text is tokenized according to the BERT base uncased tokenizer \citep{bert} and token ids are converted
to embeddings using the BERT base model. The input $\textbf{H}^{s}_{0, w} \in \mathbb{R}^{M+2\times D}$ to the
cropped BERT base model (our text encoder) is the sequence
of token embeddings summed element-wise with the positional embeddings $\textbf{T}^{pos}_{w} \in \mathbb{R}^{M+2\times D}$ of BERT.

\[
\textbf{H}^{s}_{0, w} = [\textbf{h}^{s}_{0, w, \text{\texttt{[T\_CLS]}}}, \textbf{h}^{s}_{0, w, 1}, ..., \textbf{h}^{s}_{0, w, M}, \textbf{h}^{s}_{0, w, \text{\texttt{[T\_SEP]}}}] + \textbf{T}^{pos}_{w}
\]

Here the superscript $s$ denotes that the representation stems from the student model, which will later be important for knowledge distillation.
The subscript for a single token is of the form <layer, modality, token>, where $0$ denotes the input to the first layer of the BERT base model.
Correspondingly $1$ denotes the output of the first layer and
therefore the input to the second layer, and so on. The subscript $w$ denotes the text modality.
Inspired by \citet{beit3}, we set the maximum sequence length $M$ to 64 for efficiency, which means that we
only utilize the first 64 pretraining positional embeddings of the BERT model. The special tokens \texttt{[T\_CLS]} and \texttt{[T\_SEP]}
are taken directly from the BERT base model and are also pretrained. 
The notation is inspired by \citet{vlmo}.

\textbf{Image Representation.} Each image $\mathbf{x} \in \mathbb{R}^{H\times W\times C}$ is patchified, flattened, and each resulting patch
is projected into a $D$-dimensional embedding according to \citep{vit}. The parameter used for the patch projection stem directly from the
pretrained Data2Vec2 \citep{data2vec2} image model. Since we make use of the ViT-B/16 architecture \citep{vit}, $D$ equals 768, which
also holds for the BERT base model. In all our experiments the image resolution is set to $224\times 224$ pixels. Similar to the text
representation, we define the image representation $\textbf{H}^{s}_{0, v} \in \mathbb{R}^{N+1\times D}$ as:

\[
\textbf{H}^{s}_{0, v} = [\textbf{h}^{s}_{0, v, \text{\texttt{[I\_CLS]}}}, \textbf{h}^{s}_{0, v, 1}, ..., \textbf{h}^{s}_{0, v, N}] + \textbf{T}^{pos}_{v}
\]

Here, the subscript $v$ denotes the image modality, and $N$ is the number of patches. The special token \texttt{[I\_CLS]} is taken directly
from the Data2Vec2 \citep{data2vec2} image model and is also pretrained. The positional embeddings $\textbf{T}^{pos}_{v} \in \mathbb{R}^{N+1\times D}$ are sinusoidal.

\textbf{Forward Pass} Let our cropped BERT base model be denoted by $f_{w}(\cdot)$, the cropped Data2Vec2 image model by $f_{v}(\cdot)$, and the shared
encoder by $f_{s}(\cdot)$. Each image representation $\textbf{H}^{s}_{0, v}$ and text representation $\textbf{H}^{s}_{0, w}$ is first passed through
the pretrained image and text encoder, respectively.

\[
\textbf{H}^{s}_{L, v} = f_{v}(\textbf{H}^{s}_{0, v}) \quad \text{and} \quad \textbf{H}^{s}_{L, w} = f_{w}(\textbf{H}^{s}_{0, w})
\]

Since both encoders have 6 layers, it holds that $L=6$. After being passed through the encoders, the representations
are passed separately through the shared encoder.

\[
\textbf{H}^{s}_{K, v} = f_{s}(\textbf{H}^{s}_{L, v}) \quad \text{and} \quad \textbf{H}^{s}_{K, w} = f_{s}(\textbf{H}^{s}_{L, w})
\]

Again, this is just one Transformer layer, so $K=L+1=7$. The final representations for image and text are the representations
of the \texttt{[I\_CLS]} and \texttt{[T\_CLS]} tokens. They are denoted by $\textbf{h}^{s}_{K, v, \text{\texttt{[I\_CLS]}}}$ and
$\textbf{h}^{s}_{K, w, \text{\texttt{[T\_CLS]}}}$, respectively.

\subsection{Contrastive Learning}
An approach central, but not unique to our approach is the use of a contrastive loss to align image and text. We use the contrastive loss
as presented by \citet{clip}, and follow the approach of \citet{vlmo} by gathering negative examples from all GPUs to increase the
effectiveness of the contrastive loss. We use the representations $\textbf{h}^{s}_{K, v, \text{\texttt{[I\_CLS]}}}$ and
$\textbf{h}^{s}_{K, w, \text{\texttt{[T\_CLS]}}}$ for image and text, respectively, which are normalized before computing the cosine similarity
between all possible pairs of image and text in the current batch. For equations that formulate the contrastive loss, we refer to \citet{vlmo}.

\subsection{Self-Supervised Knowledge Distillation}
What makes our approach unique is the use of knowledge distillation to guide the alignment between image and text. 
We use BEiTv2 \citep{beitv2} as the teacher model. For each image-text pair in a batch, we pass the image to BEiTv2
and extract the representations of the \texttt{[I\_CLS]} token, denoted by
$\textbf{h}^{t}_{K, v, \text{\texttt{[I\_CLS]}}}$, from the last layer. Since BEiTv2 acts as the teacher model, we add the subscript $t$ to the
representation.

\textbf{Contrastive Distillation.} Unlike \cite{shre}, our teacher is self-supervised, so we cannot use the kl-divergence loss on the probability
distributions of the ImageNet-1K \citep{imagenet} classes. Instead, we perform a contrastive loss between the student and teacher representations.
Let $f_{p}(\cdot)$ denote a projection head, which is a single linear layer. The image representations and text representations,
created by our vision-language, are passed through the projection head and normalized.

\[
\textbf{z}^{s}_{v} = ||f_{p}(\textbf{h}^{s}_{K, v, \text{\texttt{[I\_CLS]}}})||_2 \quad \text{and} \quad \textbf{z}^{s}_{w} = ||f_{p}(\textbf{h}^{s}_{K, w, \text{\texttt{[T\_CLS]}}})||_2
\]

We then perform the contrastive loss once between the image representations of the student $\textbf{z}^{s}_{v}$ and teacher $\textbf{z}^{t}_{v}$,
and once between the text representations of the student $\textbf{z}^{s}_{w}$ and the image representations of the teacher $\textbf{z}^{t}_{w}$.


\textbf{Memory Bank.}

Note that we use a self-supervised vision teacher model in order to fulfill both the criteria of
being self-supervised and independent of pretrained vision-language components.

The final training objective is given by:

\[
\min \mathcal{L}_{cl} + \mathcal{L}_{kd}
\]

\subsection{Pretraining Setup}

\section{Results}
\subsection{Image-Text Retrieval}
\subsection{Image Classification}
\subsection{Text Classification}
\subsection{Ablation Studies}
\section{Limitations and Future Work}
- mome arch
\section{Conclusion}


\subsubsection*{Broader Impact Statement}




\subsubsection*{Acknowledgments}


\bibliography{main}
\bibliographystyle{tmlr}

\appendix
\section{Appendix}
\subsection{Discussion on Curated Data} \label{curated_data}

\end{document}
